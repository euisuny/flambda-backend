\documentclass[11pt,fleqn]{amsart}
\usepackage{amssymb}
\usepackage{proofs}
\usepackage{mathpartir}
\usepackage[color=yellow]{todonotes}



\begin{document}

\title{Validator for Flambda2 Simplifier}
\date{}
\maketitle{}

%% \section{Flambda2 Simplifier}

\section{Reduction strategy}

%%  Macros for core exps
\newcommand{\letexp}[3]{\mathsf{let}\; #1 \; = \; #2 \; \mathsf{in} \; #3}
\newcommand{\letcodeexp}[3]{\mathsf{let}\; (\mathsf{code} \; #1) \; = \; #2 \; \mathsf{in} \; #3}
\newcommand{\subst}[3]{#1 \; [ #2 \setminus #3 ]}
\newcommand{\letcont}[4]{#1 \; \mathsf{where} \; #2 \; #3 \; = \; #4}
\newcommand{\step}{\longrightarrow}
\newcommand{\applycont}[2]{#1 \; #2}
\newcommand{\apply}[4]{#1 \; \, #2 \; \, #3 \; \, #4}
\newcommand{\abst}[2]{\lambda \; #1.\, #2}
\newcommand{\switch}[2]{\mathsf{switch}\; (#1) \; #2}
\newcommand{\args}[1]{\overrightarrow{#1}}


This language has a call-by-value style reduction strategy.
Notice the unusual [LetR] rule---the expression $N$ refers to an expression in
the normal form, which may refer to a normalized effectful expression.
This rule is \textit{not} analogous to the [ApplyContR] rule, since the
lambda abstraction is always implicit in let expressions, ensuring that the
``lefthand-side'' of the application is always a value. This is necessary
because for the case of several effectful expressions (such as a print
statement), inlining the occurence of the expression multiple times will be
behaviorally different from the original expression.

The [Apply-$\beta$] rule describes the case when the callee is a lambda
expression, and the argument is fully evaluated.
The expression is beta-reduced, then the resulting value get passed on as an
argument to either the return or exception continuation, depending on whether
or not the expression throws an exception.

\begin{mathpar}\footnotesize
  \inferrule
      [\footnotesize Let-$\beta$]
      {}
      {\letexp{x}{v}{e} \step \subst{e}{x}{v}}

  \inferrule
      [\footnotesize LetL]
      {e_1 \step e_1'}
      {\letexp{x}{e_1}{e_2} \step \letexp{x}{e_1'}{e_2}}

  \inferrule
      [\footnotesize LetR]
      {e_2 \step e_2'}
      {\letexp{x}{N}{e_2} \step \letexp{x}{N}{e_2'}}

  \inferrule
      [\footnotesize LetClo-$\beta$]
      {}
      {\letexp{(\mathsf{clo} \; x)}{K}{e} \step \subst{e}{x}{(x, K)}}

  \inferrule
      [\footnotesize LetCode-$\beta$]
      {}
      {\letcodeexp{x}{v}{e} \step \subst{e}{x}{v}}


  %% \inferrule
  %%     [\footnotesize LetCodeDeleted]
  %%     {}
  %%     {\letcodeexp{x}{\mathsf{Deleted}}{e} \step e}

  %% \inferrule
  %%     [\footnotesize LetCodeNewer]
  %%     {}
  %%     {\letcodeexp{(\mathsf{newer\_of} \; x) \; x'}{e_1}{e_2} \step
  %%         \letcodeexp{x}{\subst{e_1}{x'}{x}}{\subst{e_2}{x'}{x}}}

  %% IY: This rule is not great because the lifting of statics is not uniform:
  %% we don't want to implement this, the reasoning is too finnicky
  %% \inferrule
  %%     [\footnotesize LetStaticClo]
  %%     {e_2 \step e_2'}
  %%     {\letcodeexp{\mathcal{K}^{\uparrow}}{e_1}{e_2} \step \letcodeexp{\mathcal{K}^{\uparrow}}{e_1}{e_2'}}

  %% \inferrule
  %%     [\footnotesize LetClo]
  %%     {}
  %%     {\letcodeexp{\mathcal{K}}{e_1}{e_2} \step \letcodeexp{\mathcal{K}^{\uparrow}}{e_1^{\uparrow}}{\subst{e_2}{\mathcal{K}}{\mathcal{K}^{\uparrow}}}}

  %% \inferrule
  %%     [\footnotesize LetCode]
  %%     {e_2 \step e_2'}
  %%     {\letcodeexp{x}{N}{e_2} \step \letcodeexp{x}{N}{e_2'}}

  \inferrule
      [\footnotesize LetCont-$\beta$]
      {}
      {\letcont{e_1}{k}{\args{args}}{e_2} \step \subst{e_1}{k}{\abst{\args{args}}{e_2}}}


  \inferrule
      [\footnotesize ApplyContR]
      {\args{args} \step \args{args}'}
      {\applycont{v}{\args{args}} \step \applycont{v}{\args{args}'}}

  \inferrule
      [\footnotesize ApplyContL]
      {k \step k'}
      {\applycont{k}{\args{args}} \step \applycont{k'}{\args{args}}}

  \inferrule
      [\footnotesize ApplyCont-$\beta$]
      {}
      {\applycont{(\abst{x}{e})}{v} \step \subst{e}{x}{v}}

  \inferrule
      [\footnotesize ApplyR]
      {\args{args} \step \args{args}'}
      {\apply{e}{res_k}{exn_k}{\args{args}} \step \apply{e}{res_k}{exn_k}{\args{args}'}}

  \inferrule
      [\footnotesize ApplyL]
      {e \step e'}
      {\apply{e}{res_k}{exn_k}{\args{args}} \step \apply{e'}{res_k}{exn_k}{\args{args}}}

  \inferrule
      [\footnotesize Apply-$\beta$]
      {}
      {\apply{(\abst{\args{x}}{e})}{res_k}{exn_k}{\args{v}} \step \subst{e}{x}{\args{v}} \hookrightarrow res_k/exn_k}

  \inferrule
      [\footnotesize Switch]
      {e \step e'}
      {\switch{e}{arms} \step \switch{e'}{arms}}

  \inferrule
      [\footnotesize Switch-$\beta$]
      {}
      {\switch{v}{[x \mapsto e]} \step \subst{e}{x}{v}}

      \overrightarrow{args} \step \overrightarrow{args}' := \forall a, a \in \overrightarrow{args}. \exists a', a \step^\ast a' \land a \in \overrightarrow{args}'
\end{mathpar}

\section{Rewrite Rules}

\begin{mathpar}
  \inferrule
      [FlattenMatch]
      {}
      {\switch{\switch{e_1}{[A \mapsto e_2 : B | .. ]}}{[B \mapsto e_2' | .. ]} \step \switch{e_1}{[A \mapsto \subst{e_2'}{B}{e_2} | .. ]}}
\end{mathpar}


\section{Features}

A wishlist of desirable inlining/semantic features to support for the validator.

\subsection{Inlining}
\begin{itemize}
\item function calls
\item recursive functions
\item inlining (direct calls, within same function)
\item cross-module inlining
\item low-priority: locals
\end{itemize}

\subsection{Semantics}
\begin{itemize}
\item mutable state
\item exceptions
\item effects (printing, etc.)
\item external calls
\end{itemize}

\subsection{Primitives evaluation}
\begin{itemize}
\item arithmetic evaluation: commutative and associative laws for arithmetic?
  It is likely that the commutative/associative laws are not necessary for the simplifier
\item block-based primitives (makeblock, loading from block)
  The blocks have a tag, corresponding to the constructors (i.e. tag0 is the first constructor)
  values either are immediate tags or blocks
  Mutability corresponds to reference cells

  Being able to treat the block-related primitives will resolve supporting the structures below
  (except for arrays, which have a tricky case involving storing floating-point values.
  See floating-point valued array optimization)
\end{itemize}
TODO: Refactor [simplify\_primitive].

\subsection{Supported structures}
\begin{itemize}
\item structs
\item tuples
\item lists
\item arrays
\end{itemize}

\end{document}
